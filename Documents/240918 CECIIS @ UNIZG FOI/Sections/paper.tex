\section{Introduction}

\begin{frame}
    \frametitle{Introduction}
    \begin{itemize}
        \item Digital transformation is a thorough transformation of business through digital technologies.
        \item Critical in this process: accurately describing the business process to be digitized.
        \item Digital twins (DT) serve as intermediate products in transformation, acting as testbeds.
        \item This paper introduces agent-based simulation models for richer feedback in the transformation process.
    \end{itemize}
\end{frame}

\begin{frame}
    \frametitle{Problem Definition}
    \begin{itemize}
        \item Traditional token-based simulations provide limited feedback for business process transformations.
        \item Proposal: Integrate ontology-based agent models into intelligent virtual environments (IVE).
        \item The addition of gamification can enhance feedback quality by simulating more nuanced, real-world interactions.
    \end{itemize}
\end{frame}

\begin{frame}
    \frametitle{Related Work}
    \begin{itemize}
        \item \textbf{Intelligent Agents}: Systems capable of perceiving and acting upon their environment.
        \item \textbf{Intelligent Virtual Environments (IVE)}: Combine high-fidelity environmental simulations with AI for real-time interaction.
        \item \textbf{Digital Twins}: Real-time synchronizations between virtual and physical systems.
        \item Applications include predictive maintenance and dynamic process optimization in Industry 4.0.
    \end{itemize}
\end{frame}

\section{Proposed Architecture}

\begin{frame}
    \frametitle{Proposed Architecture}
    \begin{itemize}
        \item \textbf{Three-Phase Architecture:}
        \begin{enumerate}
            \item \textbf{Setup Phase}: Map real-world elements to the virtual environment using an ontology.
            \item \textbf{Implementation Phase}: Translate ontology-based models into simulation-ready blueprints.
            \item \textbf{Execution Phase}: Use data to simulate agent behaviors and observe feedback.
        \end{enumerate}
        \item Agents are implemented as digital twins, and their behavior is observed using different levels of integration (Digital Model, Digital Shadow, Digital Twin).
    \end{itemize}
\end{frame}

\section{Use Cases}

\begin{frame}
    \frametitle{Use Case: Production Planning}
    \begin{itemize}
        \item \textbf{Production Planning and Scheduling Process:}
        \begin{enumerate}
            \item Planning: Scheduling raw materials and resources.
            \item Preparation: Gathering necessary documents and determining current production states.
            \item Scheduling: Assigning resources and determining when tasks are executed.
            \item Implementation: Executing production tasks and adjusting based on real-time feedback.
        \end{enumerate}
        \item The proposed architecture simulates the impact of digital transformation in these phases.
    \end{itemize}
\end{frame}

\begin{frame}
    \frametitle{Scenario 1: Planning and Preparation}
    \begin{itemize}
        \item Digital inputs (material stocks, work orders) are synchronized in real-time.
        \item Production plans dynamically adjust to changes in the market and inventory.
        \item Simulation models customer orders, inventory, and suppliers as interacting agents.
    \end{itemize}
\end{frame}

\begin{frame}
    \frametitle{Scenario 2: Scheduling and Implementation}
    \begin{itemize}
        \item Work orders are allocated digitally to available resources.
        \item Real-time updates provide continuous feedback on capacity and workload.
        \item Simulation adjusts production plans based on incidents like equipment failures.
    \end{itemize}
\end{frame}

\section{Discussion and Conclusion}

\begin{frame}
    \frametitle{Discussion}
    \begin{itemize}
        \item Simulating digital transformation using agents offers richer, more interactive feedback than traditional methods.
        \item Incorporating gamification techniques mimics real-world human behavior changes.
        \item Applications of AI in business process modeling create more accurate transformation insights.
    \end{itemize}
\end{frame}

\begin{frame}
    \frametitle{Conclusion and Future Work}
    \begin{itemize}
        \item Contributions: A three-phase simulation framework using agents, ontologies, and gamification.
        \item Future Research:
        \begin{itemize}
            \item Expanding ontology for business processes.
            \item Integrating machine learning for more advanced decision-making.
            \item Validating the framework with empirical case studies.
        \end{itemize}
    \end{itemize}
\end{frame}
