\begin{frame}
    \frametitle{Introduction}
    \begin{itemize}
        \item The paper focuses on gamification techniques applicable to improving \alert{group decision-making} in shared transportation.
        \item Gamification often targets individual users; however, group dynamics are crucial in optimising resource use.
        \item Key elements: cooperation, competition, and social gamification.
    \end{itemize}
\end{frame}

\begin{frame}
    \frametitle{Gamification Concepts}
    \begin{itemize}
        \item \alert{Gamification} is about using game elements in non-game contexts to enhance engagement.
        \item Common elements: avatars, badges, leaderboards, challenges.
        \item In \alert{shared transportation}, the challenge is aligning group decisions for resource optimisation.
    \end{itemize}
\end{frame}

\section{Social Gamification: Cooperation and Competition}
\begin{frame}
    \frametitle{Social Gamification}
    \begin{itemize}
        \item \alert{Social} Gamification: Enhancing individual tasks through group interaction and cooperation.
        \item \alert{Competition and cooperation} trigger different psychological processes.
    \end{itemize}
\end{frame}

\begin{frame}
    \frametitle{Gamification Categories}
    \begin{itemize}
        \item Four types of gamification based on \alert{social dependence}:
        \begin{itemize}
            \item Individual-based: Focus on individual performance.
            \item Competitive: Users compete for rewards (e.g., leaderboards).
            \item Cooperative: Teams work toward shared goals.
            \item Competitive-cooperative: Hybrid of cooperation within teams and competition between teams.
        \end{itemize}
        
        \item Cooperative settings result in higher participation and foster feelings of belonging.

        \item \alert{Coopetition} enhances engagement, group cohesion, and overall system performance.
    \end{itemize}
\end{frame}

\begin{frame}
    \frametitle{Implications of Social Gamification}
    \begin{itemize}
        \item Psychological Impact: Cooperative environments enhance user well-being and \alert{engagement}.
        \item Social Identity: Belonging to a group increases motivation and commitment.
        \item Effective gamification design leverages both competitive and cooperative dynamics to motivate users.
    \end{itemize}
\end{frame}

\section{Use Case \& the Proposed Approach}

\begin{frame}{Use Case}

    \begin{itemize}
        \item We consider the user of the \alert{transport system} $i$, represented by his travel request: $$p_i(O_i, D_i, T_I)$$

        \item If the transport operator detects that this request is too expensive for the system to fulfil or simply impossible, the gamification system is activated to motivate \alert{the user to modify it}. 
        
        \item The system will calculate the possible modifications and present them to the user together with a \alert{reward that motivates} them to accept a suggested modification. 
    \end{itemize}

\end{frame}

\begin{frame}
    \frametitle{Group Gamification Approach}
    \begin{itemize}
        \item The proposed system \alert{merges cooperation and competition}.
        \item Groups are divided into:
        \begin{itemize}
            \item Guilds: Long-term groups that cooperate and compete against other guilds.
            \item Parties: Short-term groups formed to solve immediate challenges.
        \end{itemize}
    \end{itemize}
\end{frame}

\begin{frame}
    \frametitle{Key Elements of the Proposed Approach}
    \begin{itemize}
        \item Individual vs. Group Dynamics:
        \begin{itemize}
            \item Individual users compete for rewards but must cooperate within their party.
            \item Guilds compete at an intergroup level while parties cooperate to maximise rewards.
        \end{itemize}
        \item Motivation Mechanisms:
        \begin{itemize}
            \item Leaderboards, badges, and custom rewards drive individual engagement.
            \item Visual feedback emphasises both individual and group progress.
        \end{itemize}
    \end{itemize}
\end{frame}

\begin{frame}
    \frametitle{Benefits of the Approach}
    \begin{itemize}
        \item Enhanced Engagement:
        \begin{itemize}
            \item Combining competition and cooperation fosters deeper involvement.
            \item Temporary groupings (parties) encourage negotiation and decision-making.
        \end{itemize}
        \item Social Dynamics:
        \begin{itemize}
            \item Cooperative intergroup interactions increase team cohesion.
            \item Guilds enhance long-term participation and group loyalty.
        \end{itemize}
    \end{itemize}
\end{frame}

\section{Conclusion}
\begin{frame}
    \frametitle{Conclusion}
    \begin{itemize}
        \item Social gamification, through cooperation and competition, can enhance shared transportation systems.
        \item The proposed system uses both short-term and long-term groupings to balance individual and group dynamics.
        \item Future research will explore and validate additional gamification techniques through simulations and real-world testing.
    \end{itemize}
\end{frame}
