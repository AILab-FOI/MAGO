\chapter{Framework Design and Description}\label{ch: Framework Design}
% The initial document describing the framework's design for instantiating and running a multiagent system described using an ontology, and possibly several iterations of it, as it is being refined.

% Developed framework that allows the user to instantiate agents and run a multiagent system based on the description of the desired system using the developed ontology for describing and instantiating a multiagent system.

The main objective of the \magoontologyname framework is to provide a medium for converting a \ac{MAS} model defined using the related \magoontologyname ontology into a template for implementing the modelled system using \ac{SPADE} library of Python. The rendered implementation template is not expected to include all the details that might be needed to run the finished system of agents. Still, it is planned to provide the initial implementation requirements of the modelled system. Including all the implementation details in the ontology might prove to be too cumbersome and taxing for the modelling process.

The framework is expected to translate the necessary elements of the ontology to classes, objects, and instances where applicable and provide the rest of the ontology knowledge to agents translated into applicable data types.

\section{Framework Desgn}

The key requirements of the framework are, therefore, the following.

\begin{itemize}
    \item \mintedInline{Agent} subconcepts must be translatable into classes extending the \mintedInline{Agent} class of \ac{SPADE}. Every \ac{SPADE} agent must connect to an \ac{XMPP} server in order to be able to communicate with other agents. To connect to an \ac{XMPP} server, the agent must have a \mintedInline{name} and the address of the \ac{XMPP} server it is connecting to. Furthermore, individuals of the \mintedInline{Agent} must be translated into objects of the appropriate agent class defined in \ac{SPADE}.

    \item \mintedInline{Behaviour} individuals can usually be found in the extension of one of the six subconcepts of the \mintedInline{Behaviour} concept. These individuals must be implemented by extending the appropriate class defined in \ac{SPADE}. Since behaviour implementations highly depend on the intended use of the system and the agents therein, various details of the actual implementation of behaviour are not planned to be a part of the implementation template generated by this framework. Therefore, \mintedInline{Behaviour} individuals are expected to be translated only to the point of a defined behaviour class that can be instantiated by individual agents. One key observation is that agents, by default, know no behaviours. Instead, they learn about the available behaviours by playing, i.e. enacting, different \mintedInline{Role} individuals. Individuals of the \mintedInline{Role} concept are planned to be implemented in a way that is accessible by an agent, e.g. as a value of their internal attribute.

    \item By default, \ac{SPADE} agents communicate using the \ac{XMPP} protocol that requires a connection to an active \ac{XMPP} server. Therefore, every agent must be connected to exactly one individual of the \mintedInline{Agent Host Server} concept. This concept must provide the host the \mintedInline{Agent} individual has to connect to, while the other part of the \ac{JID}, the name, is provided by the \mintedInline{Agent} individual itself.
\end{itemize}

The \magoontologyname ontology provides the basic concepts for the framework to translate. However, the framework must be able to work with additional subconcepts introduced to the \mintedInline{Agent} concept. This requirement stems from the need to allow the system modeller to create agent classes and their individual agents. Furthermore, the \magoontologyname framework must work with individuals, even though treating individuals of \mintedInline{Agent} concept is expected to be different to how individuals of the \mintedInline{Behaviour} concept are treated; individual behaviours should be implemented as behaviour classes that will be instantiated by individual agents, while individual agents are instances of the applicable \mintedInline{Agent} class. Ultimately, extending the framework to include additional concepts that may be introduced to the related ontology in the future should not be extremely difficult.

\primjer{
    Subconcepts of the \mintedInline{Agent} concept can be \mintedInline{Agent Factory} and\\\mintedInline{Agent Recipe} in the domain where recipe agents consume a subset of the set of services provided by factory agents, which is, in turn, a subset of the system-wide set of possible services. 
}{Subconcepts to the \mintedInline{Agent} concept}

Finally, the framework should be implemented to provide the user with its functionality without requiring extensive programming or \ac{SPADE} knowledge. In other words, the framework must be easy to run and provide the results straightforwardly.



\section{Framework Description}

